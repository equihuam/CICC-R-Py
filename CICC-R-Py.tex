% Options for packages loaded elsewhere
% Options for packages loaded elsewhere
\PassOptionsToPackage{unicode}{hyperref}
\PassOptionsToPackage{hyphens}{url}
\PassOptionsToPackage{dvipsnames,svgnames,x11names}{xcolor}
%
\documentclass[
  spanish,
  letterpaper,
  DIV=11,
  numbers=noendperiod]{scrartcl}
\usepackage{xcolor}
\usepackage{amsmath,amssymb}
\setcounter{secnumdepth}{-\maxdimen} % remove section numbering
\usepackage{iftex}
\ifPDFTeX
  \usepackage[T1]{fontenc}
  \usepackage[utf8]{inputenc}
  \usepackage{textcomp} % provide euro and other symbols
\else % if luatex or xetex
  \usepackage{unicode-math} % this also loads fontspec
  \defaultfontfeatures{Scale=MatchLowercase}
  \defaultfontfeatures[\rmfamily]{Ligatures=TeX,Scale=1}
\fi
\usepackage{lmodern}
\ifPDFTeX\else
  % xetex/luatex font selection
\fi
% Use upquote if available, for straight quotes in verbatim environments
\IfFileExists{upquote.sty}{\usepackage{upquote}}{}
\IfFileExists{microtype.sty}{% use microtype if available
  \usepackage[]{microtype}
  \UseMicrotypeSet[protrusion]{basicmath} % disable protrusion for tt fonts
}{}
\makeatletter
\@ifundefined{KOMAClassName}{% if non-KOMA class
  \IfFileExists{parskip.sty}{%
    \usepackage{parskip}
  }{% else
    \setlength{\parindent}{0pt}
    \setlength{\parskip}{6pt plus 2pt minus 1pt}}
}{% if KOMA class
  \KOMAoptions{parskip=half}}
\makeatother
% Make \paragraph and \subparagraph free-standing
\makeatletter
\ifx\paragraph\undefined\else
  \let\oldparagraph\paragraph
  \renewcommand{\paragraph}{
    \@ifstar
      \xxxParagraphStar
      \xxxParagraphNoStar
  }
  \newcommand{\xxxParagraphStar}[1]{\oldparagraph*{#1}\mbox{}}
  \newcommand{\xxxParagraphNoStar}[1]{\oldparagraph{#1}\mbox{}}
\fi
\ifx\subparagraph\undefined\else
  \let\oldsubparagraph\subparagraph
  \renewcommand{\subparagraph}{
    \@ifstar
      \xxxSubParagraphStar
      \xxxSubParagraphNoStar
  }
  \newcommand{\xxxSubParagraphStar}[1]{\oldsubparagraph*{#1}\mbox{}}
  \newcommand{\xxxSubParagraphNoStar}[1]{\oldsubparagraph{#1}\mbox{}}
\fi
\makeatother


\usepackage{longtable,booktabs,array}
\usepackage{calc} % for calculating minipage widths
% Correct order of tables after \paragraph or \subparagraph
\usepackage{etoolbox}
\makeatletter
\patchcmd\longtable{\par}{\if@noskipsec\mbox{}\fi\par}{}{}
\makeatother
% Allow footnotes in longtable head/foot
\IfFileExists{footnotehyper.sty}{\usepackage{footnotehyper}}{\usepackage{footnote}}
\makesavenoteenv{longtable}
\usepackage{graphicx}
\makeatletter
\newsavebox\pandoc@box
\newcommand*\pandocbounded[1]{% scales image to fit in text height/width
  \sbox\pandoc@box{#1}%
  \Gscale@div\@tempa{\textheight}{\dimexpr\ht\pandoc@box+\dp\pandoc@box\relax}%
  \Gscale@div\@tempb{\linewidth}{\wd\pandoc@box}%
  \ifdim\@tempb\p@<\@tempa\p@\let\@tempa\@tempb\fi% select the smaller of both
  \ifdim\@tempa\p@<\p@\scalebox{\@tempa}{\usebox\pandoc@box}%
  \else\usebox{\pandoc@box}%
  \fi%
}
% Set default figure placement to htbp
\def\fps@figure{htbp}
\makeatother



\ifLuaTeX
\usepackage[bidi=basic]{babel}
\else
\usepackage[bidi=default]{babel}
\fi
% get rid of language-specific shorthands (see #6817):
\let\LanguageShortHands\languageshorthands
\def\languageshorthands#1{}


\setlength{\emergencystretch}{3em} % prevent overfull lines

\providecommand{\tightlist}{%
  \setlength{\itemsep}{0pt}\setlength{\parskip}{0pt}}



 


\usepackage{booktabs}
\usepackage{caption}
\usepackage{longtable}
\usepackage{colortbl}
\usepackage{array}
\usepackage{anyfontsize}
\usepackage{multirow}
\KOMAoption{captions}{tableheading}
\makeatletter
\@ifpackageloaded{caption}{}{\usepackage{caption}}
\AtBeginDocument{%
\ifdefined\contentsname
  \renewcommand*\contentsname{Tabla de contenidos}
\else
  \newcommand\contentsname{Tabla de contenidos}
\fi
\ifdefined\listfigurename
  \renewcommand*\listfigurename{Listado de Figuras}
\else
  \newcommand\listfigurename{Listado de Figuras}
\fi
\ifdefined\listtablename
  \renewcommand*\listtablename{Listado de Tablas}
\else
  \newcommand\listtablename{Listado de Tablas}
\fi
\ifdefined\figurename
  \renewcommand*\figurename{Figura}
\else
  \newcommand\figurename{Figura}
\fi
\ifdefined\tablename
  \renewcommand*\tablename{Tabla}
\else
  \newcommand\tablename{Tabla}
\fi
}
\@ifpackageloaded{float}{}{\usepackage{float}}
\floatstyle{ruled}
\@ifundefined{c@chapter}{\newfloat{codelisting}{h}{lop}}{\newfloat{codelisting}{h}{lop}[chapter]}
\floatname{codelisting}{Listado}
\newcommand*\listoflistings{\listof{codelisting}{Listado de Listados}}
\makeatother
\makeatletter
\makeatother
\makeatletter
\@ifpackageloaded{caption}{}{\usepackage{caption}}
\@ifpackageloaded{subcaption}{}{\usepackage{subcaption}}
\makeatother
\usepackage{bookmark}
\IfFileExists{xurl.sty}{\usepackage{xurl}}{} % add URL line breaks if available
\urlstyle{same}
\hypersetup{
  pdftitle={Relación ser humano y naturaleza},
  pdfauthor={Colegio Interinstitucional sobre Cambio Climático (CICC); Miguel Equihua Zamora},
  pdflang={es},
  colorlinks=true,
  linkcolor={blue},
  filecolor={Maroon},
  citecolor={Blue},
  urlcolor={Blue},
  pdfcreator={LaTeX via pandoc}}


\title{Relación ser humano y naturaleza}
\author{Colegio Interinstitucional sobre Cambio Climático
(CICC) \and Miguel Equihua Zamora}
\date{}
\begin{document}
\maketitle


\subsection{Objetivo}\label{objetivo}

Examinar cómo las distintas civilizaciones han interpretado y
transformado sus entornos naturales, para comprender cómo es que el
clima, como gran síntesis ecológica, ha influido en cosmovisiones,
sistemas de conocimiento, expresiones culturales y en las vicisitudes
sociales y económicas a lo largo de la historia.

\subsection{Contexto}\label{contexto}

Este es le tercer módulo del curso. En los dos módulos previos se habrán
explorado las manifestaciones visibles del cambio climático, sus
impactos multidimensionales y las bases biofísicas del clima. En este
tercer bloque analizaremos las relaciones con los eventos históricos y
expresiones culturales que surgen del vínculo entre los seres humanos y
la biosfera. Nos interesa reconocer que los procesos que son intrínsecos
a un planeta vivo y dinámico (incluido el clima), no han sido solo un
escenario pasivo para el desarrollo humano, sino que han sido y siguen
siendo, actores fundamentales que moldean culturas, economías, formas de
habitar, de ser y estar en el planeta. En este contexto, nos interesa
apreciar que la actual crisis climática, (en realidad socioambiental)
es, con toda seguridad, algo que con diferencias y semejanzas la
humanidad ya ha vivido. No se trata de algo enteramente nuevo. Para
comenzar, queremos hacer la pregunta introspectiva sobre ¿qué tanto nos
importa o afecta el clima? lo que nos llevará a indagar ¿cómo es que la
sociedad (economía, salud, bienestar) se vincula con el clima? ¿tenemos
al clíma en la médula de nuestras culturas? ¿podemos reconocer esos
signos? Si logrgamos reencontramos como entes \emph{socioecológicos},
¿Cómo nos explicamos que se haya diluido la consciencia de esta
interdependencia vital? ¿qué implicaciones tendrá esta desconexión
cultural actual? ¿valdría la pena buscar reconstruir esta conciencia
socioambiental? ¿qué ganaríamos? si no lo hacemos ¿qué perdemos?

Como marco reflexivo consideraremos nuestra práctica de transformar los
ecosistemas para hacer agricultura o construir los asentamientos en los
que vivimos. ¿Cómo es que esto nos vincula con la diversidad biológica?.
Claro está, además, que todo lo que hacemos requiere, finalmente, que
utilicemos energía. Ya sea la de nuestros músculos, la de animales
domésticos, de las corrientes de agua, del viento, del fuego, o del sol.
La historia misma de cómo obtenemos esa energía, es desde luego otra
forma como nos vínculamos con el entorno, con los ecosistemas. Las
formas como algunas sociedades obtenián energía en el pasado ¿las habrá
metido en problemas? ¿habrá provocado defradación ambiental crítica?
(Mayas, Toltecas, Egipto, Mesopotamia, Nazca,Imperio Romano) Hoy parece
que nuestras prácticas de uso de energía sí lo hastan haciendo. La
situación la provoca nuestro enorme uso de combustibles fósiles y la
degradación de los ecosistemas que dan soporte a la vida en el planeta.
Si estas son ļas causas de un peligro inminente de gran magnitud ¿no
sería lo inteligente buscar cambiar el rumbo? Es a lo que nos referimos
como trtansición energética y es también a lo que aspira la
sustentabilidad del desarrollo humano. En este bloque lo que queremos es
simplemente reflexionar sobre el contexto que podría justificar buscar
nuevos rumbos y renovar el espíritu creativo y el talento de los seres
humanos. Transformar la preparación de los profesionales uniersitarios
que diseñan y operan la producción de bienes y servicios cotidianamente.
Si lo que está ocurriendo actualmente es verdaderamente una crisis
socioambiental, estamos ante una encrusijada ¿qué rumbos deberíamos
elegir?

Queridas y queridos colegas, esos son los antecedentes de la invitación
que les hemos hecho llegar. En particular los temas que nosotros
imaginamos para ustedes, sujetos a sus precisiones y reorientaciones,
son los que se muestran en el cuadro.

\begin{table}
\fontsize{12.0pt}{14.4pt}\selectfont
\begin{tabular*}{\linewidth}{@{\extracolsep{\fill}}>{\raggedright\arraybackslash}p{\dimexpr 0.30\linewidth -2\tabcolsep-1.5\arrayrulewidth}>{\raggedright\arraybackslash}p{\dimexpr 0.50\linewidth -2\tabcolsep-1.5\arrayrulewidth}}
\toprule
tema & ponente \\ 
\midrule\addlinespace[2.5pt]
 Cultura, clima y uso de la energía & Dra. Martha Micheline Cariño Olvera \\ 
 Importancia del clima en la cosmovisión del desarrollo de las culturas & Dr. Fabio Flores Granados \\ 
 Conocimiento tradicional, agricultura y biodiversidad & Dra. Rosa María González Amaro \\ 
 Conocimiento tradicional, agricultura y biodiversidad & Dra. Ana Isabel Moreno Calles \\ 
 Interdependencia naturaleza y sociedad & Dra. Verónica Vázquez García \\ 
 Transformación de los entornos (ciudades y comunidades) & Dr. José Manuel Maass Moreno \\ 
\bottomrule
\end{tabular*}
\end{table}

A manera de estampas, les comparto algunas referencias que pueden ser
evocadoras para lo que les proponemos construir entre todos.

\href{https://www.nexos.com.mx/?p=32923}{El trabajo de Florescano},
especialmente su estudio sobre los precios del maíz, resulta enormemente
revelador de como ese vínculo entre los ciclos climáticos de largo
plazo, afectan la economía y la estructura misma de las sociedades.
¿Clima y hambre?

¿Qué piensan de la noción nahuatl de \textbf{altépetl}? Por lo que
entiendo su formación implicaba la interacción con los vecinos con
quienes se competía por el control de un territorio que era
relativamente escaso, así como por los recursos naturales de los
diferentes ecosistemas que existían en él, lo mismo que por el control
de las redes comerciales, por el dominio militar y por el reconocimiento
de su legitimidad política.

¿Ven en el bajo relieve del Rey (Chalcatzingo, Morelos) algo
profundamente evocativo para nuestro tema?

\pandocbounded{\includegraphics[keepaspectratio]{../../CICC-UV/CICC-módulo-3/img/rey-de-la-cueva.jpg}}

La
\href{https://historia.nationalgeographic.com.es/a/pequena-edad-hielo-ola-frio-que-asolo-viejo-continente_18751\%5D}{\textbf{pequeña
edad de hielo}}, se refiere a la ola de frío qué asoló a Europa y Asia
principalmente, entre los siglos XVI y XVIII (quizás hasta el XIX). Se
vivieron allá condiciones climáticas con un gran descenso de las
temperaturas. Dejó huellas tanto en la vida social como en la economía
de todo el continente.

\begin{figure}[H]

{\centering \pandocbounded{\includegraphics[keepaspectratio]{../../CICC-UV/CICC-módulo-3/img/Les_Très_Riches_Heures_du_duc_de_Berry_février.jpg}}

}

\caption{Del manuscrito iluminado \emph{Très Riches Heures du Duc de
Berry}, hecho probablemente entre 1412 y 1416}

\end{figure}%

\subsection{Tareas solicitadas}\label{tareas-solicitadas}

\subsection{\texorpdfstring{\emph{Estampas} para
inspirar}{Estampas para inspirar}}\label{estampas-para-inspirar}

\href{../../CICC-UV/CICC-módulo-3/Referencias/539_R_08_historicidad_c.pdf}{Los
Altépetl}
\href{../../CICC-UV/CICC-módulo-3/Referencias/clima\%20durante\%20vida\%20de\%20Miguel\%20de\%20Cervantes.pdf}{¿Cómo
eran las condiciones climáticas en la vida de Miguel decervantes?}
\href{../../CICC-UV/CICC-módulo-3/Referencias/Dialnet-ElAltepetlNahuaComoPaisaje-7425612.pdf}{El
altepetl nahua como paisaje}
\href{../../CICC-UV/CICC-módulo-3/Referencias/El\%20Rey\%20de\%20las\%20entrañas\%20de\%20la\%20tierra\%20al\%20cielo.\%20(Chalcatzingo,\%20Morelos)\%20-\%20Ark.\%20Experiencias.pdf}{El
Rey de las entrañas de la tierra al cielo. (Chalcatzingo, Morelos)}
\href{../../CICC-UV/CICC-módulo-3/Referencias/Global\%20Crisis\%20of\%20the\%20Seventeenth\%20Century.pdf}{The
Global Crisis of the Seventeenth Century}
\href{../../CICC-UV/CICC-módulo-3/Referencias/Guerra-clima-catástrofe.pdf}{Guerra,
clima y catástrofe: una reconsideración de la crisis general del siglo
XVII y de la decadencia de España}
\href{../../CICC-UV/CICC-módulo-3/Referencias/Moore-extracted-text-La-crisis-climatica-es-una-lucha-de-clases-2021-July-Jacobin-Latin-America.pdf}{La
crisis climática es una lucha de clases}

\subsection{Referencias}\label{referencias}




\end{document}
